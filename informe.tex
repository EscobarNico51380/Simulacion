\documentclass{article}


\usepackage{arxiv}

\usepackage[utf8]{inputenc} % allow utf-8 input
\usepackage[T1]{fontenc}    % use 8-bit T1 fonts
\usepackage{hyperref}       % hyperlinks
\usepackage{url}            % simple URL typesetting
\usepackage{booktabs}       % professional-quality tables
\usepackage{amsfonts}       % blackboard math symbols
\usepackage{nicefrac}       % compact symbols for 1/2, etc.
\usepackage{microtype}      % microtypography
\usepackage{lipsum}
\usepackage{graphicx}
\graphicspath{ {./images/} }


\title{Trabajo practico 1.1 Simulacion}


\author{
 Renzo Aimaretti \\
  \texttt{renzoceronueve@gmail.com} \\
  %% examples of more authors
   \And
 Facundo Sosa Bianciotto \\
  \texttt{ZIL50@pitt.edu} \\
  \And
 Vittorio Maragliano\\
  \texttt{maraglianovittorio@gmail.com} \\
    \And
 Ignacio Amelio Ortiz\\
  \texttt{maraglianovittorio@gmail.com} \\
    \And
 Nicolás Roberto Escobar\\
  \texttt{maraglianovittorio@gmail.com} \\
    \And
 Juan Manuel De Elia\\
  \texttt{maraglianovittorio@gmail.com} \\
  %% \AND
  %% Coauthor \\
  %% Affiliation \\
  %% Address \\
  %% \texttt{email} \\
  %% \And
  %% Coauthor \\
  %% Affiliation \\
  %% Address \\
  %% \texttt{email} \\
  %% \And
  %% Coauthor \\
  %% Affiliation \\
  %% Address \\
  %% \texttt{email} \\
}

\begin{document}
\maketitle
\begin{abstract}
Este informe presenta el desarrollo e implementación de una simulación computacional del funcionamiento de una ruleta, como introducción a los conceptos básicos de la materia Simulación. El modelo fue implementado en Python 3 e incluye generación de números aleatorios, estructuras de datos, análisis estadístico y visualización de resultados mediante gráficos. Se realizaron múltiples corridas del experimento para analizar el comportamiento probabilístico del sistema, contrastar los resultados con las expectativas teóricas y evaluar la aleatoriedad y el sesgo del modelo. El objetivo principal es familiarizarse con herramientas y conceptos esenciales de simulación estocástica aplicados a un sistema simple y ampliamente conocido.

\end{abstract}


% keywords can be removed
%\keywords{First keyword \and Second keyword \and More}


\section{Introduction}
En el marco de la materia Simulación, este trabajo tiene como objetivo desarrollar una primera aproximación al estudio de sistemas aleatorios mediante la construcción e implementación de un modelo simple: una ruleta. A través de este ejemplo clásico de juego de azar, se busca comprender cómo se puede simular el comportamiento de un sistema estocástico utilizando programación.

En particular, se simulará la corrida de una ruleta tipo europea, compuesta por 37 números (del 0 al 36), replicando el proceso de selección aleatoria que ocurre en cada giro. El modelo permitirá definir distintos parámetros de entrada, como la cantidad de tiradas o el número apostado, y generará datos que serán analizados estadísticamente.

Durante el desarrollo, se emplearán herramientas del lenguaje Python 3.x, incluyendo funciones de generación de números aleatorios, estructuras de control, listas, y bibliotecas para visualización como \texttt{Matplotlib}. Además, se calcularán frecuencias absolutas y relativas, y se compararán los resultados obtenidos con los valores esperados desde un enfoque probabilístico.

Este trabajo no sólo busca practicar conceptos teóricos, sino también aplicar herramientas computacionales para analizar el comportamiento de un sistema sencillo, sentando las bases para simulaciones más complejas en trabajos futuros.


\section{Task description and data construction}
\label{sec:headings}

\paragraph{Descripcion del programa.}
El programa recibe 3 parametros para poder ejecutarse: 
\\-e: El numero elegido. 
\\-n: La cantidad de tiradas. 
\\-c: La cantidad de corridas. 
\\El programa simula el lanzamiento de una ruleta europea, que tiene 37 numeros (0-36). En cada tirada, se genera un numero aleatorio entre 0 y 36, y se verifica si coincide con el numero elegido por el usuario. Si coincide, se suma 1 al contador de aciertos. Al finalizar las tiradas, se imprime la cantidad de aciertos y la probabilidad de acertar.
\\
\begin{verbatim}
codigo simular ruleta
\end{verbatim}

\subsection{Headings: second level}
\lipsum[5]
\begin{equation}
\xi _{ij}(t)=P(x_{t}=i,x_{t+1}=j|y,v,w;\theta)= {\frac {\alpha _{i}(t)a^{w_t}_{ij}\beta _{j}(t+1)b^{v_{t+1}}_{j}(y_{t+1})}{\sum _{i=1}^{N} \sum _{j=1}^{N} \alpha _{i}(t)a^{w_t}_{ij}\beta _{j}(t+1)b^{v_{t+1}}_{j}(y_{t+1})}}
\end{equation}

\subsubsection{Headings: third level}
\lipsum[6]

\paragraph{Paragraph}
\lipsum[7]

\section{Examples of citations, figures, tables, references}
\label{sec:others}
\lipsum[8] \cite{kour2014real,kour2014fast} and see \cite{hadash2018estimate}.

The documentation for \verb+natbib+ may be found at
\begin{center}
  \url{http://mirrors.ctan.org/macros/latex/contrib/natbib/natnotes.pdf}
\end{center}
Of note is the command \verb+\citet+, which produces citations
appropriate for use in inline text.  For example,
\begin{verbatim}
   \citet{hasselmo} investigated\dots
\end{verbatim}
produces
\begin{quote}
  Hasselmo, et al.\ (1995) investigated\dots
\end{quote}

\begin{center}
  \url{https://www.ctan.org/pkg/booktabs}
\end{center}


\subsection{Figures}
\lipsum[10] 
See Figure \ref{fig:fig1}. Here is how you add footnotes. \footnote{Sample of the first footnote.}
\lipsum[11] 

\begin{figure}
  \centering
  \fbox{\rule[-.5cm]{4cm}{4cm} \rule[-.5cm]{4cm}{0cm}}
  \caption{Sample figure caption.}
  \label{fig:fig1}
\end{figure}

\begin{figure} % picture
    \centering
    \includegraphics{test.png}
\end{figure}

\subsection{Tables}
\lipsum[12]
See awesome Table~\ref{tab:table}.

\begin{table}
 \caption{Sample table title}
  \centering
  \begin{tabular}{lll}
    \toprule
    \multicolumn{2}{c}{Part}                   \\
    \cmidrule(r){1-2}
    Name     & Description     & Size ($\mu$m) \\
    \midrule
    Dendrite & Input terminal  & $\sim$100     \\
    Axon     & Output terminal & $\sim$10      \\
    Soma     & Cell body       & up to $10^6$  \\
    \bottomrule
  \end{tabular}
  \label{tab:table}
\end{table}

\subsection{Lists}
\begin{itemize}
\item Lorem ipsum dolor sit amet
\item consectetur adipiscing elit. 
\item Aliquam dignissim blandit est, in dictum tortor gravida eget. In ac rutrum magna.
\end{itemize}


\bibliographystyle{unsrt}  
%\bibliography{references}  %%% Remove comment to use the external .bib file (using bibtex).
%%% and comment out the ``thebibliography'' section.


%%% Comment out this section when you \bibliography{references} is enabled.
\begin{thebibliography}{1}

\bibitem{kour2014real}
George Kour and Raid Saabne.
\newblock Real-time segmentation of on-line handwritten arabic script.
\newblock In {\em Frontiers in Handwriting Recognition (ICFHR), 2014 14th
  International Conference on}, pages 417--422. IEEE, 2014.

\bibitem{kour2014fast}
George Kour and Raid Saabne.
\newblock Fast classification of handwritten on-line arabic characters.
\newblock In {\em Soft Computing and Pattern Recognition (SoCPaR), 2014 6th
  International Conference of}, pages 312--318. IEEE, 2014.

\bibitem{hadash2018estimate}
Guy Hadash, Einat Kermany, Boaz Carmeli, Ofer Lavi, George Kour, and Alon
  Jacovi.
\newblock Estimate and replace: A novel approach to integrating deep neural
  networks with existing applications.
\newblock {\em arXiv preprint arXiv:1804.09028}, 2018.

\end{thebibliography}


\end{document}
