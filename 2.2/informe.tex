\documentclass{article}
\usepackage[utf8]{inputenc}
\usepackage[T1]{fontenc}
\usepackage[spanish]{babel}
\usepackage{hyperref}
\usepackage{float}
\usepackage{url}
\usepackage{booktabs}
\usepackage{amsfonts}
\usepackage{amsmath}
\usepackage{nicefrac}
\usepackage{microtype}
\usepackage{graphicx}
\usepackage{caption}
\usepackage{listings}
\usepackage{color}
\graphicspath{{./images/}}
\usepackage{lmodern}
\usepackage[margin=2.5cm]{geometry}
\usepackage{fancyvrb}

\title{Trabajo Práctico 2.2 \\
Generadores Pseudoaleatorios de Distribuciones de Probabilidad}
\author{
    Renzo Aimaretti \\ \texttt{renzoceronueve@gmail.com}
    \and
    Facundo Sosa Bianciotto \\ \texttt{facundososabianciotto@gmail.com}
    \and
    Vittorio Maragliano \\ \texttt{maraglianovittorio@gmail.com}
    \and
    Ignacio Amelio Ortiz \\ \texttt{nameliortiz@gmail.com}
    \and
    Nicolás Roberto Escobar \\ \texttt{escobar.nicolas.isifrro@gmail.com}
    \and
    Juan Manuel De Elia \\ \texttt{juanmadeelia@gmail.com}
}
\date{Mayo 2025}

\begin{document}

\maketitle

\begin{abstract}
Este trabajo desarrolla la implementación de generadores de números pseudoaleatorios para diversas distribuciones de probabilidad, abordando tanto distribuciones continuas como discretas. Cada distribución se fundamenta teóricamente, se implementa computacionalmente en Python y se testea mediante herramientas visuales y estadísticas. Se utilizan métodos como la transformada inversa y el método de rechazo, conforme a los lineamientos clásicos expuestos por Thomas Naylor en su obra \textit{Técnicas de Simulación en Computadoras}.
\end{abstract}

\section{Introducción}
La generación de números pseudoaleatorios que sigan una distribución de probabilidad específica es un aspecto clave en simulación computacional. A partir de un generador uniforme confiable, se pueden construir generadores para cualquier distribución mediante distintas transformaciones. En este trabajo se presentan los generadores para distribuciones seleccionadas, junto a su justificación teórica, construcción algorítmica y evaluación empírica.

\section{Distribuciones Continuas}

\subsection{Distribución Uniforme}
\textbf{Densidad:}
\begin{equation}
f(x) = \frac{1}{b-a}, \quad a \leq x \leq b
\end{equation}

\textbf{Transformada Inversa:}
\begin{equation}
x = a + (b-a)u, \quad u \sim U(0,1)
\end{equation}

\textbf{Método de Rechazo:} Aunque la transformada inversa es directa y eficiente, también se implementó el método de rechazo. Se tomó una función constante mayor o igual que $f(x)$ (es decir, la propia constante de la densidad) y se aceptaron los valores $x$ generados dentro del intervalo $[a,b]$ con probabilidad proporcional a $f(x)$. Dado que $f(x)$ es constante, todos los valores fueron aceptados.

\textbf{Resultados:} Con $a=2$, $b=5$ y 10.000 muestras:
\begin{itemize}
\item Media empírica: 3.4951 (teórica: 3.5)
\item Varianza empírica: 0.7568 (teórica: 0.75)
\end{itemize}

\subsection{Distribución Exponencial}
\textbf{Densidad:}
\begin{equation}
f(x) = \lambda e^{-\lambda x}, \quad x \geq 0
\end{equation}

\textbf{Transformada Inversa:}
\begin{equation}
x = -\frac{1}{\lambda} \ln(u), \quad u \sim U(0,1)
\end{equation}

\textbf{Método de Rechazo:} Se utilizó una cota mayor $M$ sobre $f(x)$, y se generaron candidatos $x$ de una distribución uniforme en $[0, b]$ (por ejemplo, $b = 10$), con $u \sim U(0,1)$. Se aceptó $x$ si $u < \frac{f(x)}{M}$. El valor de $M$ fue tomado como $\lambda$, el valor máximo de la función en $x=0$.

\textbf{Resultados:} Con $\lambda = 1.5$ y 10.000 muestras:
\begin{itemize}
\item Media empírica: 0.6657 (teórica: 0.6667)
\item Varianza empírica: 0.4532 (teórica: 0.4444)
\end{itemize}

\subsection{Distribución Normal}
\textbf{Densidad:}
\begin{equation}
f(x) = \frac{1}{\sigma \sqrt{2\pi}} e^{-\frac{(x - \mu)^2}{2\sigma^2}}
\end{equation}

\textbf{Método de Box-Muller:} Se generaron dos variables $u_1, u_2 \sim U(0,1)$ y se aplicó:
\begin{equation}
z = \sqrt{-2 \ln u_1} \cos(2\pi u_2), \quad x = \mu + \sigma z
\end{equation}

\textbf{Método de Rechazo:} También se implementó el método de rechazo utilizando como propuesta una distribución uniforme acotada en $[-5,5]$ y como cota $M = \frac{1}{\sqrt{2\pi}}$. Se generó $x$ uniforme y $u \sim U(0,1)$, aceptando si $u < \frac{f(x)}{M}$.

\textbf{Resultados:} Con $\mu=0$ y $\sigma=1$:
\begin{itemize}
\item Media empírica: -0.0062 (teórica: 0)
\item Desviación estándar empírica: 1.0003 (teórica: 1)
\end{itemize}

\section{Distribuciones Discretas}

\subsection{Distribución Binomial}
\textbf{Probabilidad:}
\begin{equation}
P(X = k) = \binom{n}{k} p^k (1-p)^{n-k}
\end{equation}

\textbf{Método de Bernoulli:} Se generaron $n$ ensayos de Bernoulli con probabilidad $p$, sumando los éxitos.

\textbf{Método de Rechazo:} Se generó un valor $x$ entre $0$ y $n$ y un número uniforme $u$. Se aceptó $x$ si $u < \frac{P(x)}{M}$, con $M = \max P(x)$ precomputado.

\textbf{Resultados:} Con $n=10$, $p=0.5$:
\begin{itemize}
\item Media empírica: 5.012 (teórica: 5)
\item Varianza empírica: 2.48 (teórica: 2.5)
\end{itemize}

\subsection{Distribución Poisson}
\textbf{Probabilidad:}
\begin{equation}
P(X = k) = \frac{\lambda^k e^{-\lambda}}{k!}
\end{equation}

\textbf{Algoritmo de Knuth:} Se acumularon productos de variables uniformes hasta que el resultado fue menor que $e^{-\lambda}$.

\textbf{Método de Rechazo:} Se generó un valor $k$ natural en un intervalo razonable (por ejemplo, $[0,15]$) y se aceptó si $u < \frac{P(k)}{M}$ con $M$ estimado como el máximo de $P(k)$.

\textbf{Resultados:} Con $\lambda=4$:
\begin{itemize}
\item Media empírica: 3.995 (teórica: 4)
\item Varianza empírica: 3.92 (teórica: 4)
\end{itemize}

\subsection{Distribución Empírica}
\textbf{Método:} Se utilizó la transformación de la función de distribución acumulada (CDF). Para cada valor $u \sim U(0,1)$, se asignó un valor $x_i$ tal que $F(x_i-1) < u \leq F(x_i)$.

\textbf{Método de Rechazo:} Se propuso un valor $x$ del conjunto definido y se aceptó con probabilidad $P(x)/M$, donde $M = \max P(x)$.

\textbf{Resultados:} Las muestras obtenidas replicaron con precisión la distribución definida.

\section{Conclusión}
Se logró implementar generadores para diversas distribuciones continuas y discretas utilizando la transformada inversa, el método de rechazo y otros algoritmos clásicos. Las muestras fueron validadas mediante estadísticas de primer y segundo orden, y contrastadas con sus modelos teóricos. Incluso en casos donde el método de rechazo no es necesario, se aplicó para cumplir con los requerimientos del trabajo práctico.

\section*{Referencias}
\begin{itemize}
\item Naylor, T.H. (1982). \textit{Técnicas de simulación en computadoras}.
\item Ross, S.M. (2006). \textit{Simulación}.
\item Documentación oficial de Numpy y Scipy.
\end{itemize}

\end{document}

